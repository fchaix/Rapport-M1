% Je ne sais plus à quoi servent ces lignes. C'est sans doute une histoire de
% marge et de taille de la police des notes dans la marge.
\setlength{\marginparwidth}{3.5cm}
\let\oldmarginpar\marginpar
\renewcommand\marginpar[1]{\-\oldmarginpar[\raggedleft\footnotesize #1]
{\raggedright\footnotesize #1}}

% Création d'un environement pour les anecdotes.
% TODO : En créer d'autres (remarques, exemples, démonstrations...) et les
% mettre dans un fichier à insérer en input plutôt que de surcharger celui-là.
\newenvironment{anecdote}
   {\begin{quote} \small}
   {\normalsize \end{quote}}
% environement d’exemple :
\newenvironment{exemple}
   {\begin{quote} \small}
   {\normalsize \end{quote}}

% Une commande qui sert pour le moment à rien, pour stocker le nom du prof
\newenvironment{rmq}{$\Rightarrow$}{$\diamondsuit$}
\newcommand{\prof}[1]{}

\newcommand{\esp}[1]{\textit{#1}} % Commande pour stocker les noms d’espèces.
\newcommand{\gene}[1]{\textit{#1}} % Commande pour stocker les noms de gènes.



\newcommand{\definition}[2][1]{\paragraph{\textsc{#1}~:} #2}

% Commande pour mettre du texte en indice
\makeatletter
\DeclareRobustCommand*\down[1]{%
\@textsubscript{\selectfont#1}}
\def\@textsubscript#1{%
{\m@th\ensuremath{_{\mbox{\fontsize\sf@size\z@#1}}}}}
\makeatother
% Pareil pour l’exposant, sauf que c’est plus simple.
\newcommand{\up}[1]{\textsuperscript{#1}}

