%%%%%%%%%%%%%%%%%%%%%%%%%%%%%%%%%%%%%%%%%%%%%%%%%%%%%%%%%%%%%%%
% Important : Ces sources sont prévues pour être compilées avec XeTeX.
% Testé avec Texlive 2012.
% Encodage du texte en UTF-8
% Auteur : François Chaix <francois.chaix@etu.univ-lyon1.fr>
% Vous pouvez faire ce que vous voulez avec ce document et ce code, je le place
% sous la licence CC0 (https://creativecommons.org/publicdomain/zero/1.0/)
%%%%%%%%%%%%%%%%%%%%%%%%%%%%%%%%%%%%%%%%%%%%%%%%%%%%%%%%%%%%%%%

\documentclass[a4paper, 12pt, linktocpage=true, oneside]{memoir}
%\renewcommand{{\baselinestretch}}{1.5}
\renewcommand{\baselinestretch}{1.5}
\usepackage{polyglossia}
	\setdefaultlanguage{french}

\usepackage{geometry}
\geometry{hmargin=2.5cm,vmargin=2.5cm}

% \setmainfont[Mapping=tex-text]{Linux Libertine}
\usepackage{libertine}
% \renewcommand{\familydefault}{\sfdefault} % Pour utiliser une fonte sans empâtements (sans serif) par défaut (biolinum avec le package libertine)
\usepackage{lipsum}
\usepackage[%
    hidelinks=true
    colorlinks=false
    pdfauthor={François Chaix},
    pdftitle={Rapport de stage M1 François Chaix},
    pdfsubject={Dynamique des éléments transposables chez l'endosymbionte Wolbachia},
    pdfkeywords={Wolbachia, Transposon-display, Drosophila, parasite, endoparasite, symbiose, Leptopilina heterotoma},
    pdfproducer={XeteX, avec le package hyperref},
    pdfcreator={Xelatex}]{hyperref}
%\usepackage[sorting=none, backend=biber, style=nature]{biblatex}
%\bibliography{bib}

% Le style bibliographique AEM a été pris ici : https://github.com/roey-angel/BIBEMME
% Merci à Roey-Angel pour ce partage.
\usepackage[style=aem, natbib=true, backend=biber]{biblatex}
\addbibresource{bib.bib}

%\usepackage[Bjornstrup]{fncychap} % Ça c'était pour quand j'utilisais la classe report
%\chapterstyle{hangnum} % DUne marge un peu trop grande pour les titres. Ce qu'il faut pas faire pour grater des pages, avec cette limitation de m****...
\chapterstyle{article}
\setsecnumdepth{subsection}

\newenvironment{encart} 
   {\begin{quote} \footnotesize \renewcommand{\baselinestretch}{0}} 
   {\normalsize \end{quote}} 

%\input{pagedegarde}
% Je ne sais plus à quoi servent ces lignes. C'est sans doute une histoire de
% marge et de taille de la police des notes dans la marge.
\setlength{\marginparwidth}{3.5cm}
\let\oldmarginpar\marginpar
\renewcommand\marginpar[1]{\-\oldmarginpar[\raggedleft\footnotesize #1]
{\raggedright\footnotesize #1}}

% Création d'un environement pour les anecdotes.
% TODO : En créer d'autres (remarques, exemples, démonstrations...) et les
% mettre dans un fichier à insérer en input plutôt que de surcharger celui-là.
\newenvironment{anecdote}
   {\begin{quote} \small}
   {\normalsize \end{quote}}
% environement d’exemple :
\newenvironment{exemple}
   {\begin{quote} \small}
   {\normalsize \end{quote}}

% Une commande qui sert pour le moment à rien, pour stocker le nom du prof
\newenvironment{rmq}{$\Rightarrow$}{$\diamondsuit$}
\newcommand{\prof}[1]{}

\newcommand{\esp}[1]{\textit{#1}} % Commande pour stocker les noms d’espèces.
\newcommand{\gene}[1]{\textit{#1}} % Commande pour stocker les noms de gènes.



\newcommand{\definition}[2][1]{\paragraph{\textsc{#1}~:} #2}

% Commande pour mettre du texte en indice
\makeatletter
\DeclareRobustCommand*\down[1]{%
\@textsubscript{\selectfont#1}}
\def\@textsubscript#1{%
{\m@th\ensuremath{_{\mbox{\fontsize\sf@size\z@#1}}}}}
\makeatother
% Pareil pour l’exposant, sauf que c’est plus simple.
\newcommand{\up}[1]{\textsuperscript{#1}}


\title{Étude des transferts horizontaux de l’endosymbionte \textit{Wolbachia} par la méthode du Transposon Display}
\author{François \textsc{Chaix}}
%%%%%%%%%%%%%%%%%%%%%%%%%%%%%%%%%%%%%%%%%%%%%%%%%%%%%%%%%%%%%%%%
\begin{document}
\maketitle
\tableofcontents

\chapter{Introduction}
% \esp{Wolbachia} est une bactérie endosymbiotique%
% \footnote{Dans ce mémoire, nous utiliseront le terme de symbiose dans son sens le plus large, a savoir regroupant à la fois les relations de symbioses au sens strict, mais aussi le parasitisme, et les différentes décinaisons se situant entre ces deux notions. Le point commun entre ces notions est le carractère obligatoire pour le symbiote.}
% obligatoire au spectre d’infection très large, infectant une grande diversité d’insectes phylogénétiquement éloignés, et même certains nématodes.

% On pourrait s’aendre, de part son carractère obligatoire, à ce que cee symbiose ait fait subir au génome de Wolbachia le même type de contraintes évolutives que celles subies par les autres bactéries endosymbiotiques décrites dans la liérature, contraintes aboutissant généralement à une réduction drastique de la taille du génome, à la perte de beaucoup de gènes liés à la survie sous forme libre, et à la perte des éléments répétés dans ce génome.

% Cependant, même si certaines de ces carractéristiques se retrouvent dans le génome de Wolbachia, d’autres ne se vérifient pas, et c’est notement le cas des éléments répétés, habtuellement perdus par les bactéries endosymbiotiques, mais que l’on retrouve en un nombre anormalement important pour une bactérie de cette écologie.
% Note : Ne pas parler tout de suite de Wolbach.

L’edosymbiose%
\footnote{Dans ce mémoire, nous utiliseront le terme de symbiose dans son sens le plus large, a savoir regroupant à la fois les relations de symbioses au sens strict, mais aussi le parasitisme, et les différentes décinaisons se situant entre ces deux notions. Le point commun entre ces notions est le carractère obligatoire pour le symbiote.\\
On peut noter au passage que \textit{Wolbachia} se situe justement sur cette barrière floue, étant parasite (au sens strict) dans beaucoup de cas, mais parfois mutualste, en offrant une protection contre les virus chez certains hôtes.}
obligatoire est une écologie colplexe, impliquant des mécanismes très différents que ceux que l’on a l’habitude d’étudier dans les autres systèmes microbiens.

Cette écologie bien particulière implique des effets assez spécifiques sur l'évolution de ces organismes, notement du point de vue de son organisation génomique.
En effet, une des premières conséquences du mode de vie intracellulaire obligatoire est une taille efficiace très réduite, dûe au carractère essentiellement horizontal de la transmission la bactérie dans le temps et dans la population.

Cette réduction drastique de la taille efficace induit un changement radical dans la répartition des pressions de séléction. Celles-ci auront tendance à ne contre-séléctionner fortement que les mutations ayant un coût très élevé.
C'est pour cela que l'on observe assez fréquement chez ces bactéries symbiotiques à transmissoin verticale des pertes de nombreux gènes\cite{wernegreen2002} dont  l'inactivation ne serait pas suffisement délétère pour être contre-séléctionné dans une contexte de faibles populations subissant des effets \textit{bottleneck} récurrents (a savoir à chaque génération de l'hôte).

% Pourquoi plus d'IS ?

Ainsi, nous pourrons noter la perte des gènes liés à a recombinaison, et l'absence, ou du moins la rareté d'éléments répétés comme l'ADN mobile des transposons, expliqué aussi par la prédominence du hasard sur la séléction dans le contexte de faibles populations sujettes à de nombreux \textit{bottlenecks}.

\paragraph{Le cas \textit{Wolbachia}\\}
Cependant, comme toute règle générale en biologie, on trouve toujours des exceptions. Dans notre cas, la bactérie endosymbiote \textit{Wolbachia} fait honneur à cette maxime pour presque tous les points. 

Tout d'abord, elle ne semble pas se transmettre uniquement de façon verticale, comme peut en témoigner la non-corrélation entre les phylogénies de celle-ci et celles de ces multiples hôtes\cite{vavre1999}.

Ensuite, l’habituelle perte des gènes de la recombinaison, et des éléments répétés ne s’observe pas chez \textit{Wolbachia}. En effet, les études notement des souches \textit{wMel} et \textit{wRi} ont montré une fréquence de recombinaison nettement suppérieure aux autres bactéries intracellulaires obligatoires.

Par la suite, le transposon \textit{ISWpi1} a été particulièrement décrit\cite{Cordaux2008}, entre autres pour sa spécificité à \textit{Wolbachia}.%, et ce sera celui-là même que nous utiliserons comme marqueur moléculaire dans ce stage

\paragraph{Choix d’un modèle d’hôte : \textit{Leptopilina heterotoma} et \textit{Drosophila sp.}\\}
Pour mieux comprendre ces transferts horizontaux, il est judicieux de se placer dans un contexte de très forte proximité écologique entre deux hôtes de \textit{Wolbachia} philogénétiquement éloignés. Cela justifie le choix de ce couple d’hôte, ayant des relations parasitoïde‐hôte impliquant donc une forte proximité des deux organismes durant toiut leur développement.

De plus, une étude (\cite{vavre1999}) a démontré la présence de transferts récents de Wolbachia entre ces deux organismes.

\begin{encart}
\paragraph{Écologie des parasitoïdes — Mise au point sur le contexte} % (fold)
\label{par:parasitoïdes}
Les parasitoïdes sont des parasites qui se développent à l’intérieur d’autres organismes. Dans le cas de \esp{L. heterotoma}, la femelle pond dans les jeunes larves de drosophile (stade 1), et la larve du parasitoïde reste dans celle de la drosophile tout le temps de sa croissance, puis finit par la consommer au stade de pupe (dernière mue avant la mue imaginale).
% paragraph parasitoïdes (end)
\end{encart}

\paragraph{Transposon Display\\} % (fold)
\label{par:transposon_display}
le transfert de \esp{Wolbachia} entre \textit{Leptopilina heterotoma} et \textit{Drosophila sp.} étant très récent, les souches associées à ces hôtes sont très similaires, allant jusqu’à être presque identiques, à la base près, quand on les séquence avec les méthodes traditionnelles ne prenant pas en compte l’ordre des gènes.

Pour établir une philogénie entre ces souches, afin d’étudier ces transferts, il nous faut donc utiliser un autre marqueur moléculaire. C’est là qu’intervient le transposon \textit{ISWpi1}, déjà évoqué auparavent. En effet, il est présent chez ces souches, et ses locus d’insertion dffèrent entre elles.

La méthode du transposon-display nous servira à établir un \textit{fingerprint} de ces insertions, afin d’établir une philogénie non basée sur les séquences nucléotidiques, mais sur les locus d’insertion de \textit{ISWpi1}.
% paragraph transposon_display\ (end)

\paragraph{} % (fold)
\label{par:Sujet}
Ce stage consistera donc en un typage de nouvelles lignées de drosophiles acquises récement par le laboratoire et pas encore testées en transposon-display. Une deuxième partie du stage consistait à l’obtension de lignées \esp{L. heterotoma} présentant un statut d'infection particulier, mais nous n'en parlerons pas cans le corps de ce mémoire, dans un souci de concision\footnote{Voir annexes}.
% paragraph  (end)

\chapter{Matériel et méthodes}
\section{Échantillonnage} % (fold)
\label{sec:échantillonnage}

	Nous échantillonons des souches de \esp{Wolbachia} appartenant à des groupes intra-spécifiques identiques sur la base du gène-marqueur \textit{wsp}, un gène dont la variabiité est suffisante pour distinguer des groupes à l’échelle intra-spécifique. Ces groupes sont nommés en général en fonction de leur hôte (\esp{wMel} est la bactérie infectant la drosophile \esp{Drosophila melanogaster} par exemple).

	Notons parmi ces souches que l’on peut ensuite les regrouper dans des groupes encore plus homogènes, car contenant des bactéries identiques sur la base de la concaténation de plusieurs séquences hautement variables à la manière de \textit{wsp}.
	C’est le cas par exemple de \esp{wRi} et de \esp{wAur}, qui formeront un groupe que nous appellerons \esp{wRi-like}.
	\begin{figure}[h]
		\begin{center}
		\begin{tabular}{|c|c|c|}
			\hline
			\textbf{Hôte (souche)}			&\textbf{Déjà testées}				&\textbf{Nouvelles}\\
			\hline
			\esp{D. simulans} (wRi)	&Japon (Tokyo)				&Botswana\\
									&France (Grand Ferrade)		&Afrique du Sud\\
									&Portugal (Chicharo)		& \\
									&Mozambique (Chimoio)		& \\
			\hline
			\esp{D. melanogaster} (wMel)& USA (Seattle) 		& Botswanna (4 sites)\\
									&Pérou						&Zimbabwe\\
									&Madagascar					& \\
			\hline
			\esp{D. auraria} (wAur)	& Terrain (1 échantillon)	&Japon\\
			\hline
			\esp{D. triauraria} (wTriaur)&							&Japon\\
			\hline
			\esp{D. suzukii} (wSuz)	&							&France(6 sites)\\
									& 							&Japon\\
			\hline
			\esp{D. subpulchrella} (wSub)		&							&Japon\\
			\hline
		\end{tabular}
		\end{center}
		\caption{Tableau récapitulatif de la répartition spatiale des échantillons de drosophiles, ainsi que du nom de leurs \esp{Wolbachia} associées. En tout, 54 échantillons ont été typés en transposon-display. La colonne «déjà testées» liste la répartition des échantillons ayant déjà été testés lors des traveaux de Hélène \textsc{Henri}\cite{memHH}, avec l’ancienne version du protocole (Cf. partie \ref{}), et la colonne «nouvelle» correspond aux nouvelles lignées testées lors de ce stage, avec la nouvelle version du protocole.}
		\label{fig:tab1}
	\end{figure}

	L’échantillonnage a été réalisé sur de multiples souches, issues de lignées de drosophiles provenant de plusieurs origines géographiques (voir tableau \ref{fig:tab1}), afin d’avoir une vue de la variabilité de notre marqueur tant sur une distrubution basée sur la phylogénie de l’hôte que sur un plan spatial.

	% \paragraph{\textit{L. heterotoma}} % (fold)
	% \label{par:hetero_mm}
	% \textit{L. heterotoma} étant dans la nature infectées par trois souches distinctes de \esp{Wolbachia}, il faut, pour typer celles-ci en Transposon-display, obtenir des lignées mono-infectées. Pour cela, nous appliquons une méthode basée sur des traitements antibiotiques ménagés, que nous décrivons en annexe, non décrits dans ce rapport.
	% % paragraph hetero_mm (end)

% subsection échantillonnage (end)

\section{Extractions d'ADN et statuts d'infection} % (fold)
	Deux protocoles d'extractions d'ADN ont été suivies : 
	Pour la vérification des statuts d'infection, les extractions ont été faites en suivant le protocole Chelex®, méthode peu coûteuse mais produisant un matériel non purifié.

	L'efficacité de l'extraction a été vérifiée par le biais d'une PCR sur le gène \gene{Its2}, gène mitochondrien généraliste.

	Pour les besoins en précision du protocole du transposon-display, des extractions d'ADN produisant de l'ADN purifié ont dû être pratiqués, en suivant le protocole du kit \textit{NucleSpin® Tissue} de l’entreprise Macherey-Nagel.

	Les statuts d'infection ont été déterminés par PCR sur la base du gène \gene{FtsZ}\footnote{Amorces : FtsZ-F2/FtsZ-R2}.

	Pour certains échantillons de drosophiles, l'espèce exacte n'était pas déterminée de façon certaine sur la base de la morphologie. Elle a donc été déterminée sur la base du séquençage du produit de la PCR du gène mitochondrial \gene{COI}\footnote{Amorces~: LCO/HCO}, à partir des échantillons extraits au Chelex®.

	Puis, la caractérisation de la souche de la bactérie a été effectuée sur la base du gène \gene{wsp}\footnote{Amorces : 81F/2R}, connu pour sa variabilité au sein du genre \esp{Wolbachia}, fort utile pour le typage de la souche.

	Enfin, la présence du transposon \textit{ISWpi1} a été testée par PCR avec les amorces ISWpi1-F/ISWpi1-R.

\section{Transposon-display} % (fold)
\label{sec:transposon_display}

	\paragraph{Principe général} % (fold)
	\label{par:principe_TnDisp}
	Le transposon-display est une méthode de biologie moléculaire visant à établir un \textit{fingerprint} des locus d’insertion d’un transposon. 
	Elle est tirée de la méthode ASAP (Allele-Specific Alu PCR)\cite{ASAP}.
	Nous allons décrire ici brièvement son principe (voir le schéma \ref{fig:figure1})~:
	\begin{enumerate}
		\item Le génome est digéré par une enzyme de restriction (ici, HindIII), séléctionné pour couper fréquemment, mais pas dans le transposon.
		\item Des adaptateurs\footnote{Dessinés par Hélène \textsc{Henri}, décrits dans son mémoire \cite{memHH}} sont ensuite appliqués à cet ADN digéré, conçus pour être d’un côté complémentaires au motif de coupure de HindIII, et de l’autre pour ne pas être complémentaire sur la fin, afin de former un "Y" et d’éviter ainsi la circularisation.
		\item Nous nous retrouvons donc avec une série de fragments linéaires d’ADN de l’échantillon, flanqués de notre adaptateur, contenant pour certains le transposon qui nous intéresse (\textit{ISWpi2}).
		\item L’étape finale consiste en deux PCR, avec comme amorce commune une amorce spécifique à l’adaptateur (La partie en "Y"), et l’autre soit à la partie 5’ du transposon (ISb), soit la partie 3’ (ISc).
	\end{enumerate}
	% Le résultat (renvoyer à figure gels) de cette PCR, après éléctrophorèse, nous donne un \textit{barcode} dans lequel chaque bande correspond à un locus d’insertion du transposon.
	Après migration sur gel d’éléctrophorèse, nous obtenons un profil dans lequel chaque fragment correspond à un locus d’insertion du transposon.
	% paragraph principe_TnDisp (end)

	\paragraph{Des nouveautés dans le protocole} % (fold)
	\label{par:protocole2}
	Ce stage a été l’occasion pour tester une nouvelle version du protocole du transposon-display, en particulier au niveau de la PCR, avec l’utilisation d’une Taq-polymérase haute fidélité (AccuTaq de la société Sigma), et l’ajustement en conséquence du programme de thermocycleur.
	% paragraph protocole2 (end)

\begin{figure}[h!]
	\begin{center}
		\includegraphics[width=160mm]{tdisplay.png}
	\end{center}
	\caption{Principe du Transposon Display avec amplification sélective des régions
flanquantes des insertions. Figure povenant du mémoire de Hélène \textsc{Henri}\cite{memHH}.% Là j'aimerais bien footciter, mais y'a un bug connu quand on footnote des trucs depuis une caption :(
	}
	\label{fig:figure1}
\end{figure}

% subsection transposon_display (end)


\chapter{Résultats \& Discussion}
\section{Résultats} % (fold)
\label{sec:résultats}

\paragraph{Polymorphisme.} % (fold)
\label{par:polymorphisme}
Nous retrouvons des profils similaires à ceux déjà obtenus auparavent\cite{memHH}, enrichis de nouvelles lignées pour les espèces déjà typées en transposon-display, et d’espèces typées pour la première fois, comme \esp{D. auraria}, \esp{D. triauraria} et \esp{D. suzukii}.
La figure \ref{fig:profils} montre un exemple de chacun de ces profils, sachant qu’il n’y avait aucune variabilité visible au sein de ceux-ci, sauf pour \esp{wMel}, qui présente un fragment de plus % Nombre de pb ?
dans certaines populations.

\begin{figure}[tb]
	\begin{center}
		\includegraphics[width=150mm]{images/profils_crop.png}
	\end{center}
	\caption{Reconstruction d'un gel représentant tous les profils carractéristiques des souches de \esp{Wolbachia} en transposon-display avec les amorces Isb/LNP.}
	\label{fig:profils}
\end{figure}

% paragraph polymorphisme (end)

\paragraph{Une amélioration nette du protocole.} % (fold)
\label{par:proto}
Un des éléments notables dans les gels d'éléctrophorèse obtenus est la nette amélioration de l'amplification des fragments de grande taille, notamment pour les profils de type \textit{wMel} (Cf. Figure \ref{fig:wMelcomp}). 
\begin{figure}[tb]
	\begin{center}
		\includegraphics[width=80mm]{images/wMel_comp.png}
	\end{center}
	\caption{Comparaison des deux protocoles sur wMel (\esp{Wolbachia} de \esp{D. melanogaster})~:
	A1 et A2 : Ancien protocole\cite{memHH}~;
	B1 et B2 : Nouveau protocole, avec accuTaq}
	\label{fig:wMelcomp}
\end{figure}
% paragraph proto (end)

% section r_sultats (end)

\section{Conclusion \& Perspectives} % (fold)
\label{sec:ccl}
Afin de tirer des conclusions complètes sur les tranferts horizontaux, il faut encore attendre les résultats de transposon-display sur des lignées de \esp{L. heterotoma} infectées par une seule souche de \esp{Wolbachia}.
Nous ne possédons pas encore ces lignées, mais une seconde partie de ce stage, non développée ici; a consisté à démarrer un traitement antibiotique ménagé sur des \esp{L. heterotoma} tri-infectées (statut d'infection sauvage), afin d'obtenir des lignées présentant un statut d'infection différent pour enfin les typer en transposon-display.

La suite de ce travail consistera à séquencer les fragments obtenus afin de situer précisément les insertions dans le génome, afin d’éventuellement en tirer des conclusions sur les conséquences physiologiques des différentes insertions du transposon. Peut-être trouverons-nous là une piste pour expliquer l’adaptabilité exceptionnelle de \esp{Wolbachia} aux changements d’hôtes ?
% section conclusion_&_perspectives (end)


\printbibliography
\listoffigures

\newpage
\pagestyle{empty}
\renewcommand{\baselinestretch}{1} % Pour enlever l'interligne 1.5
\begin{abstract}
Ceci est un résumé qu'il est trop bien.

À imprimer à part, pour l'intégrer à la quatrième de couverture.

Peut-être le compiler à part, dans un autre fichier. On verra.
\end{abstract}
\end{document}
