% \esp{Wolbachia} est une bactérie endosymbiotique%
% \footnote{Dans ce mémoire, nous utiliseront le terme de symbiose dans son sens le plus large, a savoir regroupant à la fois les relations de symbioses au sens strict, mais aussi le parasitisme, et les différentes décinaisons se situant entre ces deux notions. Le point commun entre ces notions est le carractère obligatoire pour le symbiote.}
% obligatoire au spectre d’infection très large, infectant une grande diversité d’insectes phylogénétiquement éloignés, et même certains nématodes.

% On pourrait s’aendre, de part son carractère obligatoire, à ce que cee symbiose ait fait subir au génome de Wolbachia le même type de contraintes évolutives que celles subies par les autres bactéries endosymbiotiques décrites dans la liérature, contraintes aboutissant généralement à une réduction drastique de la taille du génome, à la perte de beaucoup de gènes liés à la survie sous forme libre, et à la perte des éléments répétés dans ce génome.

% Cependant, même si certaines de ces carractéristiques se retrouvent dans le génome de Wolbachia, d’autres ne se vérifient pas, et c’est notement le cas des éléments répétés, habtuellement perdus par les bactéries endosymbiotiques, mais que l’on retrouve en un nombre anormalement important pour une bactérie de cette écologie.
% Note : Ne pas parler tout de suite de Wolbach.

L’endosymbiose%
\footnote{Dans ce mémoire, nous utiliseront le terme de symbiose dans son sens le plus large, a savoir regroupant à la fois les relations de symbioses au sens strict, mais aussi le parasitisme, et les différentes décinaisons se situant entre ces deux notions. Le point commun entre ces définitions reste le carractère obligatoire pour le symbiote.\\
On peut noter au passage que \textit{Wolbachia} se situe justement sur cette barrière floue, étant parasite (au sens strict) dans beaucoup de cas, mais parfois mutualiste, en offrant une protection contre les virus chez certains hôtes.}
obligatoire est une écologie complexe, impliquant des mécanismes très différents que ceux que l’on a l’habitude d’étudier dans les autres systèmes microbiens.

Cette écologie bien particulière implique des effets assez spécifiques sur l'évolution de ces organismes, notement du point de vue de son organisation génomique.
En effet, une des premières conséquences du mode de vie intracellulaire obligatoire est une taille efficiace de la population très réduite, dûe au carractère essentiellement vertical de la transmission de la bactérie dans le temps et dans la population.

Cette réduction drastique de la taille efficace induit un changement radical dans la répartition des pressions de séléction. Celles-ci auront tendance à ne contre-séléctionner fortement que les mutations ayant un coût très élevé.
C'est pour cela que l'on observe assez fréquement chez ces bactéries symbiotiques à transmission verticale des pertes de nombreux gènes\cite{wernegreen2002} dont  l'inactivation ne serait pas suffisement délétère pour être contre-séléctionné dans une contexte de faibles populations subissant des effets \textit{bottleneck} récurrents (a savoir à chaque génération de l'hôte).

% Pourquoi plus d'IS ?

Ainsi, nous pourrons noter la perte des gènes liés à a recombinaison, et l'absence, ou du moins la rareté d'éléments répétés comme l'ADN mobile des transposons, expliqué aussi par la prédominence du hasard sur la séléction dans le contexte de faibles populations sujettes à de nombreux \textit{bottlenecks}.

\paragraph{Le cas \textit{Wolbachia}\\}
Cependant, comme toute règle générale en biologie, on trouve toujours des exceptions. Dans notre cas, la bactérie endosymbiote \textit{Wolbachia} fait honneur à cette maxime pour presque tous les points. 

Tout d'abord, elle ne semble pas se transmettre uniquement de façon verticale, comme peut en témoigner la non-corrélation entre les phylogénies de celle-ci et celles de ces multiples hôtes\cite{vavre1999}.

Ensuite, l’habituelle perte des gènes de la recombinaison, et des éléments répétés ne s’observe pas chez \textit{Wolbachia}. En effet, les études notement des souches \textit{wMel} et \textit{wRi} ont montré une fréquence de recombinaison nettement suppérieure aux autres bactéries intracellulaires obligatoires.

Par la suite, le transposon \textit{ISWpi1} a été particulièrement décrit\cite{Cordaux2008}, entre autres pour sa spécificité à \textit{Wolbachia}.%, et ce sera celui-là même que nous utiliserons comme marqueur moléculaire dans ce stage

Pour finir, une carractéristique très importante de \esp{Wolbachia} est sa distribution au sein des arthropodes. En effet, quand la plupart des bactéries intracellulaires strictes sont pour la plupart extremement restreintes en terme de diversité d'hôtes, \esp{Wolbachia} se retrouve avec un spectre d'hôte très grand, allant des arthropodes (beaucoup de familles) aux nématodes. On estime à 66\,\% la proportion d'espèces d'invertébrés infectées par \esp{Wolbachia}\cite{hilgenboecker2008}.
Cette forte prévalence et ses spécificités génomiques sont-elles liées ? De plus en plus d'éléments semblent aller dans ce sens. Et c'est en gardant en tête cette hypothèse que ce stage essaiera d'étudier les transferts horizontaux de cette bactérie.

\paragraph{Choix d’un modèle d’hôte : \textit{Leptopilina heterotoma} et \textit{Drosophila sp.}\\}
Pour mieux comprendre ces transferts horizontaux, il est judicieux de se placer dans un contexte de très forte proximité écologique entre deux hôtes de \textit{Wolbachia} philogénétiquement éloignés. Cela justifie le choix de ce couple d’hôte, ayant des relations parasitoïde‐hôte impliquant donc une forte proximité des deux organismes durant toiut leur développement.

De plus, une étude (\cite{vavre1999}) a démontré la présence de transferts récents de Wolbachia entre ces deux organismes.

\begin{encart} % Ça crée une erreur à la compilation, de mettre un \paragraph dans un encart. Pas le temps de chercher d’où ça vient. Un simple <enter> skipe l’erreur.
	\paragraph{Écologie des parasitoïdes — Mise au point sur le contexte} % (fold)
	\label{par:parasitoïdes}
	Les parasitoïdes sont des parasites qui se développent à l’intérieur d’autres organismes. Dans le cas de \esp{L. heterotoma}, la femelle pond dans les jeunes larves de drosophile (stade 1), et la larve du parasitoïde reste dans celle de la drosophile tout le temps de sa croissance, puis finit par la consommer au stade de pupe (dernière mue avant la mue imaginale).
	% paragraph parasitoïdes (end)
\end{encart}

\paragraph{Transposon Display\\} % (fold)
\label{par:transposon_display}
le transfert de \esp{Wolbachia} entre \textit{Leptopilina heterotoma} et \textit{Drosophila sp.} étant très récent, les souches associées à ces hôtes sont très similaires, allant jusqu’à être presque identiques, à la base près, quand on les séquence avec les méthodes traditionnelles ne prenant pas en compte l’ordre des gènes.

Pour établir une philogénie entre ces souches, afin d’étudier ces transferts, il nous faut donc utiliser un autre marqueur moléculaire. C’est là qu’intervient le transposon \textit{ISWpi1}, déjà évoqué auparavent. En effet, il est présent chez ces souches, et ses locus d’insertion dffèrent entre elles.

La méthode du transposon-display nous servira à établir un \textit{fingerprint} de ces insertions, afin d’établir une philogénie non basée sur les séquences nucléotidiques, mais sur les locus d’insertion de \textit{ISWpi1}.
% paragraph transposon_display\ (end)

\paragraph{} % (fold)
\label{par:Sujet}
Ce stage consistera donc en un typage de nouvelles lignées de drosophiles acquises récement par le laboratoire et pas encore testées en transposon-display. Une deuxième partie du stage consistait à l’obtension de lignées \esp{L. heterotoma} présentant un statut d'infection particulier, mais nous n'en parlerons pas cans le corps de ce mémoire, dans un souci de concision\footnote{Voir annexes}.
% paragraph  (end)S