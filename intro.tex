L’endosymbiose obligatoire est une écologie complexe, impliquant des mécanismes évolutifs très différents de ceux que l’on rencontre habituellement dans les autres systèmes microbiens.
Les génomes de ces bactéries suivent une voie souvent similaire au cours de l’évolution, menant à une réduction de la taille du génome, par la perte de nombreux gènes\cite{wernegreen2002} notamment ceux qui sont impliqués dans le processus de recombinaison.
Il en découle donc un génome où les éléments répétés sont pratiquement absents (ou, du mois, rares\cite{Bordenstein2005}).

Cependant, la bactérie endosymbiote \esp{Wolbachia} ne suit pas cette tendance. En effet, les études des souches \esp{wMel} et \esp{wRi} notamment ont montré une fréquence de recombinaison nettement suppérieure aux autres bactéries intracellulaires obligatoires\cite{Wu2004}.

D’autre part, une carractéristique très importante de \esp{Wolbachia} est la grande variété de ses hôtes. En effet quand la plupart des bactéries intracellulaires strictes sont extrêmement restreintes en terme de diversité d’hôtes, \esp{Wolbachia} présente un spectre d’hôtes très grand, allant des arthropodes (beaucoup de familles) aux nématodes. On estime à 66\,\% la proportion d’espèces d’invertébrés infectés par \esp{Wolbachia}\cite{hilgenboecker2008}.

Pour finir, il a été mis en évidence \cite{vavre1999} des transferts horizontaux de la bactérie entre deux hôtes phylogénétiquement distants.
%, notamment quand l’écologie de ces hôtes implique une forte proximité physique de ceux-ci (dans cette étude, les deux hôtes \esp{Leptopilina heterotoma} et \esp{Drosophila sp.} entretiennent une relation parasitoïde/parasité).

Ces propriétés concervées de recombinaison et cette capacité à s’adapter à des hôtes très différents sont bien sûr à mettre en relation.
% En effet, la recombinaison implique une certaine variabilité, et on peut facilement faire un lien enre variabilité et capacité d’adaptation. 
Mais avant de proposer des hypothèses sur les causes de cette capacité d’adaptation, il est nécesaire de décrire ces événements de transfert.

Les transferts les plus récents sont mis en évidence par une similarité entre deux \esp{Wolbachia} issus d’hôtes différents, sur la base de gènes habituellement suffisement variables pour différencier deux souches intra-spécifiques.
Dans ce cas-là, lorsque sur la base des séquences connues pour être très variables, il est impossible de différencier deux bactéries, il faut se baser sur un carractère polymorphe nous permettant une résolution plus fine.

C’est dans cette optique que, durant ce stage, nous allons nous intéresser à l’étude  d’un marqueur non plus nucléotidique, mais basé sur le locus d’insertion d’un transposon, \esp{ISWpi1}, avec l’outil du transposon-display, développé par Hélène \textsc{henri}\cite{memHH}.