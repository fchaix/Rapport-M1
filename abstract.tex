Le but de ce stage est d'étudier la diversité d'un polymorphisme chez la
bactérie endosymbiote \esp{Wolbachia}, les locus d'insertion d'un transposon,
dans un contexte micro-évolutif, c'est à dire entre des souches identiques du
point de vue des séquences, mais dont l'ordre des gènes est différent.
Nous utiliserons pour cela la technique du transposon-display, méthode de
\textit{fingerprinting} permettant d'obtenir un profil témoignant de la
variabilité des locus d'insertion de cet ADN mobile.

Au cours de ce stage, nous caractériserons au moyen de cette technique des
\textit{Wolbachia} issues de lignées déjà testées auparavant en transposon-%
display, mais aussi de nouvelles lignées, de diverses provenances
géographiques. Nous en profiterons pour modifier certains points de ce
protocole, pour en améliorer la précision.
