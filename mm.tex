\subsection{Échantillonnage} % (fold)
\label{sub:échantillonnage}

	\paragraph{Drosophiles} % (fold)
	\label{par:drosophiles_mm}
	Nous échantillonons des souches de \esp{Wolbachia} ayant comme point commun d'être toutes très semblables à la souche \textit{wRi} infectant \esp{Drosophila simulans}, d'un point de vue nucléotidique. On parle de souches \emph{wRi-like}.

	\begin{figure}[h]
		\begin{center}
		\begin{tabular}{|c|c|c|}
			\hline
				&Déjà testées&Nouvelles&
			\hline
			\esp{D. simulans} (wRi)&Japon (Tokyo)&Botswana\\
				&France (Grand Ferrade)&Afrique du Sud\\
				&Portugal (Chicharo)& \\
				&Mozambique (Chimoio)& \\
			\hline
			\esp{D. melanogaster} (wMel)& USA (Seattle) & Botswanna (4 sites)\\
				&Pérou&Zimbabwe\\
				&Madagascar& \\
			\hline
			\esp{D. auraria} (wAur)& Terrain (1 échantillon)&1 lignée\\
			\hline
			\esp{D. suzukii}&&France(6 sites)\\
				& &Japon\\
			\hline
		\end{tabular}
		\end{center}
		\caption{Tableau récapitulatif du plan d'échatillonnage.}
		\label{fig:tab1}
	\end{figure}

	Les échantillons proviennent donc de plusieurs espèces de drosophiles, et de plusieurs points du globe (voir tableau \ref{fig:tab1}), en vue de situer les transferts horizontaux tant dans la dimention spatiale que phylogénétique.
	% paragraph drosophiles (end)

	\paragraph{\textit{L. heterotoma}} % (fold)
	\label{par:hetero_mm}
	\textit{L. heterotoma} étant dans la nature infectées par trois souches distinctes de \esp{Wolbachia}, il faut, pour typer celles-ci en Transposon-display, obtenir des lignées mono-infectées. Pour cela, nous appliquons une méthode basée sur des traitements antiiotiques ménagés, que nous décrivons en annexe, dans un souci de concision.
	% paragraph hetero_mm (end)

	\paragraph{Extractions d'ADN et statuts d'infection} % (fold)
	\label{par:extractions_et_statuts_d_infection}
	Deux protocoles d'extractions d'ADN ont été suivies : 
	Pour la vérification des statuts d'infection, les extractions ont été faites en suivant le protocole Chelex®, méthode peu coûteuse mais produisant un mélange non purifié.
	Pour les besoins en précision du protocole du transposon-display, des extractions d'ADN prosuisant de l'ADN purifié ont dû être pratiqués, en suivant le protocole du kit Machery Nagel®.

	Les statuts d'infection ont été déterminés par PCR sur la base du gène \gene{wsp}
	% paragraph extractions_et_statuts_d_infection (end)

% subsection échantillonnage (end)

\subsection{Transposon-display} % (fold)
\label{sub:transposon_display}

% subsection transposon_display (end)