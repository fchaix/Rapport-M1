\subsection{Échantillonnage} % (fold)
\label{sub:échantillonnage}

	\paragraph{Drosophiles} % (fold)
	\label{par:drosophiles_mm}
	Nous échantillonons des souches de \esp{Wolbachia} ayant comme point commun d'être toutes très semblables à la souche \textit{wRi} infectant \esp{Drosophila simulans}, d'un point de vue nucléotidique. On parle de souches \emph{wRi-like}.

	\begin{figure}[tb]
		\begin{center}
		\begin{tabular}{c|c|c}
				&déjà testées&Nouvelles&
			\hline
			plop&plop&plop
		\end{tabular}
		\end{center}
		\caption{Tableau récapitulatif du plan d'échatillonnage}
		\label{fig:tabéch}
	\end{figure}

	Les échantillons proviennent donc de plusieurs espèces de drosophiles, et de plusieurs points du globe, en vue de situer les transferts horizontaux tant dans la dimention spatiale que phylogénétique.
	% paragraph drosophiles (end)

	\paragraph{\textit{L. heterotoma}} % (fold)
	\label{par:hetero_mm}
	\textit{L. heterotoma} étant dans la nature infectées par trois souches distinctes de \esp{Wolbachia}, il faut, pour typer celles-ci en Transposon-display, obtenir des lignées mono-infectées. Pour cela, nous appliquons une méthode basée sur des traitements antiiotiques ménagés, que nous décrivons en annexe, dans un souci de concision.
	% paragraph hetero_mm (end)

% subsection échantillonnage (end)

\subsection{Transposon-display} % (fold)
\label{sub:transposon_display}

% subsection transposon_display (end)